\documentclass[11pt]{article}
\usepackage{amsmath,amssymb}
\usepackage{graphicx}
\usepackage{geometry}
\geometry{margin=1in}

\title{Supplementary Material:\\Operational Constraints on Ethically-Weighted Quantum Measurement}
\author{Christopher Michael Baird}
\date{\today}

\begin{document}
\maketitle

\section{Higgs invisible likelihood approximation}
CMS reports a best-fit invisible branching fraction and asymmetric uncertainties for the 2012--2018 combination \cite{CMS_HIG20_003}. In the main text we approximate the one-dimensional profile likelihood ratio by an asymmetric Gaussian in the scanned parameter $B$:
\[
q(B)\approx
\begin{cases}
\left(\frac{B-B_{\hat{}}}{\sigma_-}\right)^2 & B<B_{\hat{}}\\[4pt]
\left(\frac{B-B_{\hat{}}}{\sigma_+}\right)^2 & B\ge B_{\hat{}}
\end{cases}.
\]
This reproduces the published 95\% CL upper limit to good approximation and avoids reliance on plot digitization.

\section{QRNG shot-noise scaling}
For a balanced two-outcome QRNG (Born baseline $P(1)=P(0)=1/2$) and a small log-odds shift $\eta\Delta E$, the induced probability shift is $\delta p \approx \eta\Delta E/4$. The count difference $N_1-N_0$ has standard deviation $\sqrt{N}$ at $p=1/2$, yielding $\sigma_\eta\approx 2/(\sqrt{N}\Delta E)$ as used in Eq.~(9) of the main text.

\bibliographystyle{unsrt}
\bibliography{references}
\end{document}