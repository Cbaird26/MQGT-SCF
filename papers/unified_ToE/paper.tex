\documentclass[11pt]{article}
\usepackage{amsmath,amssymb}
\usepackage{graphicx}
\usepackage{booktabs}
\usepackage{hyperref}
\usepackage{geometry}
\geometry{margin=1in}

\title{Operational Constraints on Ethically-Weighted Quantum Measurement:\\
A Multi-Channel Effective Field Theory Analysis}
\author{Christopher Michael Baird}
\date{\today}

\begin{document}
\maketitle

\begin{abstract}
We investigate a class of effective modifications to quantum measurement in which outcome probabilities are weakly biased by an auxiliary scalar label that encodes ``ethical'' or valence-like structure. We formulate the proposal in an explicitly operational way, and embed it within a conservative effective field theory (EFT) extension of the Standard Model and General Relativity that admits a clean decoupling limit. We derive likelihood-level predictions for three experimentally independent channels: (i) quantum random number generators (QRNGs), (ii) invisible Higgs decays via a Higgs-portal coupling, and (iii) short-range tests of Newtonian gravity parameterized by Yukawa deviations. Using published results for Higgs invisible branching fractions and short-range inverse-square-law tests, we obtain conservative constraints on the corresponding EFT couplings and force ranges. We emphasize transparent statistical assumptions and provide a reproducibility package (hash manifests and verification scripts) suitable for preregistration and third-party audit.
\end{abstract}

\section{Introduction}
The Born rule is the central probabilistic postulate of quantum mechanics: given a state expanded in an orthonormal measurement basis $\{ |i\rangle \}$ as $|\psi\rangle=\sum_i c_i |i\rangle$, the probability of outcome $i$ is $P(i)=|c_i|^2$. While extraordinarily successful, the Born rule is, operationally, an empirical law. A broad research program asks whether small deviations from standard measurement statistics might be detectable, or at least bounded, by experiments. Examples include dynamical collapse models and phenomenological tests of non-standard measurement rules.

This work considers a specific \emph{operational} deformation of the Born rule motivated by the hypothesis that outcomes can be weakly biased by a scalar ``valence'' label $E_i$ assigned to each outcome. The interpretation of $E_i$ is deliberately left minimal: it is a real number that can encode, in principle, any externally specified classification of outcomes. The central question is then straightforward and testable:
\begin{quote}
Do measured frequencies exhibit deviations consistent with a fixed outcome-label bias parameter?
\end{quote}

We proceed in three steps. First, we define an ethically-weighted measurement rule and derive the corresponding inference problem for QRNG data. Second, we embed the bias within a conservative EFT framework by coupling a real scalar $S$ to the Higgs sector (a standard portal construction) and constrain the portal coupling with invisible Higgs decay limits. Third, we connect a light mediator to the standard Yukawa parameterization of short-range gravity tests and incorporate laboratory constraints.

\section{Ethically-Weighted Measurement Rule}
We define the ethically-weighted Born rule:
\begin{equation}
P(i) \;=\; \frac{|c_i|^2 \, e^{\eta E_i}}{\sum_j |c_j|^2 \, e^{\eta E_j}},
\label{eq:ewborn}
\end{equation}
where $\eta$ is a real coupling parameter and $E_i\in\mathbb{R}$ is a pre-assigned outcome label. The standard Born rule is recovered for $\eta\to 0$.

For a two-outcome experiment with labels $E_1, E_0$ and baseline amplitudes $|c_1|^2=|c_0|^2=\tfrac12$,
\begin{equation}
\log\frac{P(1)}{P(0)} = \eta (E_1-E_0).
\label{eq:logodds}
\end{equation}
In the small-$\eta$ regime, $P(1)\approx \tfrac12 + \eta\Delta E/4$ with $\Delta E=E_1-E_0$.

\subsection{QRNG likelihood and sensitivity scaling}
Given $N$ independent trials with $N_1$ outcomes labeled ``1'', the likelihood is binomial:
\begin{equation}
\mathcal{L}(\eta \mid N_1,N) \propto P(1)^{N_1} [1-P(1)]^{N-N_1},
\end{equation}
with $P(1)$ computed from Eq.~\eqref{eq:ewborn}. A convenient estimator in the balanced-amplitude case is
\begin{equation}
\hat\eta \approx \frac{2(N_1-N_0)}{N \Delta E},
\end{equation}
and the binomial (shot-noise) sensitivity scales as
\begin{equation}
\sigma_\eta \approx \frac{2}{\sqrt{N}\,\Delta E}.
\label{eq:sigmaeta}
\end{equation}
Figure~\ref{fig:qrng} shows the scaling of Eq.~\eqref{eq:sigmaeta}.

\begin{figure}[t]
\centering
\includegraphics[width=0.72\linewidth]{figures/fig_qrng_sensitivity.png}
\caption{Approximate QRNG sensitivity scaling for the ethically-weighted parameter $\eta$ (balanced two-outcome case, $\Delta E=1$).}
\label{fig:qrng}
\end{figure}

\section{EFT Embedding and Collider Constraint}
We now embed the framework in a conservative EFT extension that supports collider constraints. Consider a real scalar $S$ with a Higgs-portal interaction
\begin{equation}
\mathcal{L}_{\mathrm{portal}} = - g_\phi \, S^2 \, H^\dagger H,
\label{eq:portal}
\end{equation}
where $H$ is the Standard Model Higgs doublet and $g_\phi$ is dimensionless. After electroweak symmetry breaking, $H^\dagger H = (v+h)^2/2$ generates an $hS^2$ coupling with effective vertex $\lambda_{hSS}=2g_\phi v$ (in the convention $\mathcal{L}\supset -\tfrac12 \lambda_{hSS} h S^2$).

For $m_S < m_h/2$, the Higgs partial width to invisible scalars is
\begin{equation}
\Gamma(h\to SS)=\frac{g_\phi^2 v^2}{8\pi m_h}\sqrt{1-\frac{4m_S^2}{m_h^2}}.
\label{eq:hsswidth}
\end{equation}
The invisible branching fraction is $B_{\mathrm{inv}}=\Gamma_{\mathrm{inv}}/(\Gamma_{\mathrm{SM}}+\Gamma_{\mathrm{inv}})$. We use $\Gamma_{\mathrm{SM}}\simeq 4.07~\mathrm{MeV}$ for the Standard Model Higgs total width.

\subsection{CMS HIG-20-003 constraint}
CMS reports a combined 2012--2018 limit $B(H\to \mathrm{inv})<0.18$ at 95\% CL (assuming Standard Model production) and a best-fit value $B(H\to \mathrm{inv})=0.086^{+0.054}_{-0.052}$ \cite{CMS_HIG20_003}. We approximate the published one-dimensional profile likelihood ratio with an asymmetric Gaussian in $B$:
\begin{equation}
q(B)\approx
\begin{cases}
\left(\frac{B-B_{\hat{}}}{\sigma_-}\right)^2 & B<B_{\hat{}}\\[4pt]
\left(\frac{B-B_{\hat{}}}{\sigma_+}\right)^2 & B\ge B_{\hat{}}
\end{cases},
\end{equation}
with $(B_{\hat{}},\sigma_+,\sigma_-)=(0.086,0.054,0.052)$. Figure~\ref{fig:hinvq} shows this approximation.

\begin{figure}[t]
\centering
\includegraphics[width=0.72\linewidth]{figures/fig_hinv_qB_analytic.png}
\caption{Analytic CMS profile likelihood approximation scan $q(B)$ using the reported best fit and asymmetric uncertainties for the 2012--2018 combination \cite{CMS_HIG20_003}. The dashed line marks the published 95\% CL upper limit $B=0.18$.}
\label{fig:hinvq}
\end{figure}

\subsection{Constraint on $g_\phi$}
Assuming a uniform prior on $g_\phi\in[0,0.01]$ and a light scalar mass $m_S\ll m_h/2$, we map $g_\phi\mapsto B_{\mathrm{inv}}$ via Eq.~\eqref{eq:hsswidth} and form the posterior $p(g_\phi)\propto \exp[-q(B(g_\phi))/2]$. Figures~\ref{fig:gpost} and \ref{fig:bmap} show the resulting posterior and mapping. We obtain the conservative 95\% (97.5\%) credible upper bounds
\begin{equation}
g_\phi < 6.5\times 10^{-3}\quad (6.9\times 10^{-3}),
\end{equation}
under the stated prior and likelihood approximation.

\begin{figure}[t]
\centering
\includegraphics[width=0.72\linewidth]{figures/fig_hinv_gphi_posterior_analytic.png}
\caption{Posterior for the Higgs-portal coupling $g_\phi$ under the approximate CMS likelihood and a uniform prior on $g_\phi\in[0,0.01]$.}
\label{fig:gpost}
\end{figure}

\begin{figure}[t]
\centering
\includegraphics[width=0.72\linewidth]{figures/fig_hinv_B_vs_gphi_multi.png}
\caption{Mapping from the portal coupling $g_\phi$ to $B(H\to\mathrm{inv})$ for $m_S\ll m_h/2$. The dashed line indicates the CMS 95\% CL upper limit $B=0.18$ \cite{CMS_HIG20_003}.}
\label{fig:bmap}
\end{figure}

\section{Short-Range Gravity: Yukawa Deviations}
Laboratory tests of the inverse-square law constrain additional Yukawa contributions to the Newtonian potential:
\begin{equation}
V(r)=-\frac{Gm_1m_2}{r}\left[1+\alpha\,e^{-r/\lambda}\right],
\label{eq:yukawa}
\end{equation}
where $\alpha$ is the strength relative to gravity and $\lambda$ is the force range (often $\lambda=\hbar c/m$ for a mediator of mass $m$).

Lee \emph{et al.} performed a torsion-balance test down to $52~\mu$m and report that any \emph{gravitational-strength} Yukawa interaction (i.e.\ $|\alpha|=1$) must satisfy $\lambda<38.6~\mu$m at 95\% confidence \cite{Lee2020ISL}. Figure~\ref{fig:ff} illustrates this bound. Translating $\lambda$ into a mediator mass gives
\begin{equation}
m \gtrsim \frac{\hbar c}{\lambda} \approx 5.1~\mathrm{meV}\qquad (\alpha=1\ \text{case}).
\end{equation}
More general limits $|\alpha|(\lambda)$ can be incorporated directly by digitizing the published exclusion envelope; we treat that extension as a straightforward add-on to the present analysis.

\begin{figure}[t]
\centering
\includegraphics[width=0.72\linewidth]{figures/fig_fifth_force_alpha1.png}
\caption{Illustrative short-range gravity constraint in Yukawa form: Lee \emph{et al.} require $\lambda<38.6~\mu$m for gravitational-strength interactions ($|\alpha|=1$) at 95\% confidence \cite{Lee2020ISL}.}
\label{fig:ff}
\end{figure}


\subsection{Digitized exclusion envelope}
For joint multi-channel inference, one can incorporate the full published exclusion envelope $\alpha_{\max}(\lambda)$ by digitization.
Figure~\ref{fig:ffenv} shows the digitized envelope used in the accompanying inference harness.

\begin{figure}[t]
\centering
\includegraphics[width=0.78\linewidth]{figures/fig_fifth_force_envelope_digitized.png}
\caption{Digitized short-range gravity exclusion envelope $\alpha_{\max}(\lambda)$ used for confidence-mapped likelihood construction in the inference harness.}
\label{fig:ffenv}
\end{figure}



\section{Cosmological constraint (w$_0$--w$_a$)}
To incorporate cosmological expansion constraints in a minimally assumption-heavy way, we approximate the published joint constraints on $(w_0,w_a)$ by a correlated Gaussian with mean $\mu$ and covariance $\Sigma$ inferred from a reported confidence contour. Figure~\ref{fig:cosmoellipse} illustrates the implied 68\% and 95\% ellipses.

\begin{figure}[t]
\centering
\includegraphics[width=0.70\linewidth]{figures/fig_cosmo_w0wa_ellipse.png}
\caption{Illustrative correlated-Gaussian approximation to published $(w_0,w_a)$ constraints, shown as 68\% and 95\% confidence ellipses.}
\label{fig:cosmoellipse}
\end{figure}


\section{Discussion}
The ethically-weighted measurement rule of Eq.~\eqref{eq:ewborn} is operationally well-defined once the outcome labels $E_i$ are fixed \emph{prior} to data collection. The QRNG channel provides the cleanest direct probe of the parameter $\eta$, but Eq.~\eqref{eq:sigmaeta} makes explicit that reaching extremely small values of $\eta$ is shot-noise limited unless $\Delta E$ can be operationally amplified.

Collider and fifth-force channels constrain complementary aspects of a conservative EFT embedding. Invisible Higgs decays bound the Higgs-portal coupling $g_\phi$ for light scalars, while short-range gravity bounds the range and strength of Yukawa-like deviations. Importantly, these constraints can be applied regardless of any ``ethical'' interpretation: they are constraints on the EFT degrees of freedom required by one natural embedding of the framework.

\section{Reproducibility}
A full reproducibility package accompanies this manuscript. In addition, we provide a publication-grade inference harness (v11) that replaces digitized Higgs likelihood inputs with an analytic CMS profile likelihood approximation based on the reported best fit and asymmetric uncertainties: figures are generated from explicit scripts with recorded constants; numerical outputs are summarized in a machine-readable JSON file; and a hash manifest/receipt system supports third-party verification that artifacts match published results. These materials are intended to support preregistration and minimize analytic degrees of freedom in future experimental tests.

\section{Conclusion}
We presented a conservative, operational deformation of the Born rule with a scalar outcome-label bias, embedded it in an EFT framework with a Higgs-portal scalar, and derived constraints from published collider and short-range gravity results. The resulting bounds restrict portal couplings and force ranges under transparent assumptions, while QRNG tests provide a direct route to constraining the measurement-bias parameter $\eta$ itself. The framework is thus positioned for falsifiable, preregistered experimental tests with auditable inference.

\bibliographystyle{unsrt}
\bibliography{references}
\end{document}